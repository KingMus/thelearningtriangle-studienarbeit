\section{Einleitung}

Die Einleitung der Arbeit dient dem Ziel, verständlich aufzuzeigen welche Thematik in der Arbeit behandelt wird und inwiefern dabei eine Abgrenzung zu anderen Themen stattfindet. Eine Hinführung zum Thema dient ebenfalls der Zielstellung der Arbeit.

\subsection{Trendthema Artificial Intelligence}

\begin{quote}
"`Hello Dave!"'
\end{quote}

Dieses Zitat aus dem 1968 erschienenen Film "`2001: Odyssee im Weltraum"' prägte die damalige Zeit im Verständnis von künstlicher Intelligenz. Die Handlung des Films spielt im damals zukunftsfernen Jahr 2001, in der eine Jupitermission ausgeführt werden soll. Die Maschine "`HAL"' entwickelt im Verlauf der Geschichte jedoch ein unkalkulierbares Eigenleben. Nur unter großen Schäden gelingt es dem Protagonisten David "`Dave"' Bowman, "`HAL"' zu deaktivieren. (Quelle)

Heute, viele Jahre später, ist das Szenario einer eigenständig denkenden Maschine präsenter denn je. Dies liegt vor allem daran, das seit einigen Jahren das Interesse an diesem Themenbereich enorm gestiegen ist. Spiele wie Schach und Go, die analytisches Denken erfordern, beherrschen Algorithem schon lange besser als Menschen. Bilder bearbeiten, fälschen oder sogar komplett generieren ist ebenfalls keine Herausforderung mehr. Das Potential dieser Technologie ist scheinbar nur durch die eigenen Kompetenz begrenzt. 

Umso wichtiger wird es für Entwickler, zumindest Grundwissen im Bereich "`Künstliche Intelligenz"' (auch: "`Artificial Intelligence"') aufzubauen. Betrachtet man die vorhandenen Wissenspools im Internet wird schnell klar, das es eine gewaltige Menge an Informationen gibt. Viele Websiten widersprechen sich in verschiedenen Themen, je nach Erklärstil kann ein ganz eigenes Verständnis von einem Thema aufgebaut werden. Auch wenn es unmöglich ist, diese Probleme komplett zu beseitigen, so soll durch dieses Projekt der Lernprozess im Bereich "`KI"' vereinfacht werden.

\subsection{Ziel der Arbeit}

Diese Arbeit dokumentiert die Entwicklung des Projekts "`The Learning Triangle"'. TLT soll ein Lehrspiel, im folgenden als "`Educational Game"' bezeichnet, für das Thema AI darstellen. Es soll möglich sein, als Neueinsteiger mit Entwicklungsvorkenntnissen in das Thema zu finden und konkreteres Wissen für bestimmte Themenbereiche wie beispielsweise "`Neural Networks"' zu erlangen. Dazu muss genau überlegt werden, wie das Projekt gestaltet werden kann. Es muss ebenso über die Lernmethodik nachgedacht werden. Als Ergebnis dieser Arbeit soll also ein Spiel entwickelt werden, welches durch ein noch auszuarbeitendes System sinnvoll Wissen im Bereich "`Artificial Intelligence"' vermitteln kann. Die Grundlage stellt das bereits vorhandene Projekt TLT aus dem Software Engineering Kurs, welches unter Berücksichtigung der Aufgabe weiterentwickelt wird.

\subsection{Abgrenzung}

Der Bereich "Artificial Intelligence" ist viel zu groß, um ihn in einem Format wie dem eines Educational Games verpacken zu können. Deshalb ist es wichtig, klare Grenzen zu ziehen. Der Fokus dieses Projekts liegt deshalb im Groben gesehen auf dem Teilbereich "`Machine Learning"', dem für Softwareentwickler wichtigstem Themengebiet. Auch auf feinerer Ebene muss klar aufgezeigt werden, welche Themen bearbeitet werden. Dies wird in Kapitel \ref{lernbareTechnologien} näher erläutert.

\section{Projektmanagement}

\subsection{Planungstool}

Zur Unterstützung der Planung des Projekts sowie zum Timetracking wird das Tool YouTrack als Cloud Lösung verwendet. In YouTrack können Epics, User Stories, Tasks, Bugs und viele weitere Arten von Issues angelegt werden. Diese können in Sprints organisiert und auf dem Agile Board angezeigt werden. Wird der Status eines Tasks auf "in progress" gestellt, startet automatisch ein Timer. Wird der Task abgeschlossen oder auf "waiting" gestellt, wird die Timer Zeit automatisch in den Task eingetragen. YouTrack bietet dabei überall die Möglichkeit der eigenen Konfiguration.

Für das Projekt wurden folgende Epics angelegt:

\begin{itemize}
\item Projektmanagement: In diesem Epic werden alle Projektmanagement Aufgaben angelegt.
\item Programmierung: Dieses Epic ist für alle Programmier Aufgaben
\item Studienarbeit schreiben: Alle Tasks die die geschriebene Studienarbeit betreffen, werden unter diesem Epic eingeordnet.
\end{itemize}

TODO: alle User Stories und Tasks auflisten.

\subsection{Meilensteine und Qualitätssichernde Maßnahmen}

Ein wichtiger Bestandteil der Planung sind die Meilensteine sowie die Qualitätssichernden Maßnahmen. Diese wurden in einer Tabelle festgehalten und jeweils zum Sprintwechsel analysiert. 

TODO: Tabelle einfügen

Des weiteren wurde eine Meilenstein-Trend Analyse angelegt. Diese zeigt auf einen Blick, ob das Projekt in Verzug ist, oder ob alles in Ordnung ist.

TODO: MTA einfügen.

\subsection{Risikomanagement}
Warum ist Risikomanagement wichtig? Durch Risikomanagement kann bei guter Planung das Erreichen der Projektziele abgesichert werden, sowie ein scheitern des Projekts vehindert werden. Tritt zum Beispiel ein Problem auf, auf welches schnell reagiert werden muss, ist es vorteilhaft, wenn die Situation vorher schon bedacht wurde. Auch wenn manche Probleme nicht vollständig eliminiert werden können, so kann der Schaden doch reduziert werden.

Deshalb wurden für das Projekt folgende Problemszenarien ermittelt, deren Risikofaktor anhand Auftrittswahrscheinlichkeit und erwartetem Schaden errechnet sowie mit Gegenmaßnahmen ausgestattet. Zudem gibt es für jedes Szenario einen Verantwortlichen, welcher das Risiko im Auge behält um so möglichst frühzeitig die Gegenmaßnahmen einzuleiten. 

\subsubsection{Wissensproblem - 50\% - beide}
Das größte Risiko birgen die beiden großen neuen Themengebiete Artificial Intelligence und educational game. Diese müssen für die Bearbeitung der Studienarbeit in einem gewissen Maß erarbeitet und verstanden werden. Bei schlechter Erarbeitung kann unter Umständen der Sinn des Projektes nicht erfüllt werden.

Um dem entgegen zu wirken, muss ausreichend Zeit für die Aneignung von Wissen zum Thema Artificial Intelligence, für das Erarbeiten der Lernmethodik sowie die Durchführung eingeplant werden.

\subsubsection{Klausuren - 35\% - Franziska Neumann}
Aufgrund der Erfahrung aus den letzten Jahren wird es zum Ende des Semesters oftmals stressig, da die Klausuren anstehen. Die Klausurvorbereitung nimmt dann einen großen Teil der zur Verfügung stehenden Zeit ein. Darunter kann die Studienarbeit leiden.

Deshalb sollte bei der YouTrack-Planung immer die Klausurenphase im Blick behalten werden. Große, zeitaufwendige Tasks sollten vor der stressigen Zeit erledigt werden.

\subsubsection{Planung - 25\% - Marco Müller}
Ein weiteres Risiko stellt die Planung dar. Oftmals möchte man mehr erreichen, als man kann. Die unrealistische Planung kann dann aber zu Zeitproblemen führen. 

Eine Möglichkeit realistischer zu Planen stellen dabei Retrospektiven dar. In diesen wird erarbeitet, was und wie viel in welcher Zeit erledig werden konnte. Daraufhin kann geplant werden, welche Tasks in Zukunft etwa wie viel Zeit brauchen werden.

\subsubsection{Hardware - 21\% - Marco Müller}
Um ein neuronales Netz zu trainieren braucht es viel Rechenleistung. Ein Risiko ist hierbei, dass die private Rechenleistung nicht ausreicht um die Algorithmen der künstlichen Intelligenz fristgerecht zu berechnen. 

Eine Möglichkeit dieses Problem zu umgehen, ist, die "Rechenleistung" auslagern und zum Beispiel die DHBW-Computer (mit-)zuverwenden. Hierbei muss jedoch rechtzeitig ein Bedarf bei den Ansprechpartern angemeldet werden.

\subsubsection{Unkompatible Tools - 14\% - Marco Müller}
Sobald ein Projekt größer wird, werden verschiedene Tools eingesetzt. Dabei kann es passieren, dass diese nicht zusammen passen. 

Daher sollte im vorhinein geplant werden, welche Software verwendet werden soll und geprüft werden, ob diese zusammen arbeiten. Ist dies nicht der Fall sollte nach einer alternativen Software gesucht oder ein Workaround gefunden werden.

\subsubsection{Krankheit - 12\% - Franziska Neumann}
Ein weiteres Risiko besteht darin, dass ein Teammitglied krankheitsbedingt ausfällt. 

Leider kann hiergegen nicht sehr viel mehr vorbeugend unternommen werden, als darauf zu achten, ausreichend Vitamine zu essen und auf die nötige Hygiene zu achten.

\subsubsection{Kommunikation - 12\% - Franziska Neumann}
Schlechte oder keine Kommunikation führt oftmals zu Missverständnissen und kann das Ziel des Projekts gefährden. Des Weiteren können Schwierigkeiten im Team ein gutes Vorankommen des Projekts verhindern.

Um dies zu verhindern sollten regelmäßige Treffen vereinbart werden und der Kontakt ständig über Email, Messenger, Telefon sowie vor Ort in der Uni gehalten werden.

\subsection{Sonstiges}

Überarbeitung der bisher vorhandenen SE-Dokumente?

\section{Grundlagen}

hier wird noch mehr folgen

\subsection{Was ist TheLearningTriangle (TLT)}

\subsection{Historie des Projektes TLT}

\subsection{Was ist ein Educational Game}

\subsection{Was ist AI}

hier werden die für die Arbeit notwendigen Bereiche des Themas Artificial Intelligence erläutert...

\section{Vorüberlegungen zur Gestaltung von TLT}

\subsection{Verwendete Technologien zur Erstellung von TLT}

plus Erklärung warum wir uns dafür entschieden haben

\subsection{Welche Technologien können mit TLT gelernt werden}
\label{lernbareTechnologien}

plus Erklärung warum wir uns dafür entschieden haben

\subsection{Die Lernmethodik in TLT}

wie kann man das Thema gut beibringen bzw wie kann TLT gestaltet werden um sinnvoll zu lehren...

\subsection{Strukturierung von TLT}

Gliederung? (Lektionen, Level ...)

Motivation? (Gamification, ...)

UI?

\section{Umsetzung}

ziemlich offen, folgt später detaillierter...

beinhaltet: Architektur, Klassendiagramm und weitere ähnliche wichtige Dinge

\section{Fazit}

\subsection{Evaluation der Studienarbeit}

was wurde erreicht

Zielgruppenbefragung?

\subsection{Ausblick}

\section{Literatur}

\section{Anhang}