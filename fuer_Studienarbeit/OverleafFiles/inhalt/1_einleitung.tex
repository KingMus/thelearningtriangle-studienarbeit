\chapter{Einleitung}

Die Einleitung der Arbeit dient dem Ziel, verständlich aufzuzeigen, welche Thematik in der Arbeit behandelt wird und inwiefern dabei eine Abgrenzung zu anderen Themen stattfindet. Eine Hinführung zum Thema dient ebenfalls der Zielstellung der Arbeit.

\section{Trendthema Artificial Intelligence}

\begin{quote}
"`Hello Dave!"'
\end{quote}

Dieses Zitat aus dem 1968 erschienenen Film "`2001: Odyssee im Weltraum"' prägte die damalige Zeit im Verständnis von künstlicher Intelligenz. Die Handlung des Films spielt im damals noch weit in der Zukunft liegendem Jahr 2001, in dem eine Mission, eine Reise zum Jupiter, ausgeführt werden soll. Die Maschine "`HAL"' entwickelt im Verlauf der Geschichte jedoch ein unkalkulierbares Eigenleben. Nur unter großen Schäden gelingt es dem Protagonisten David "`Dave"' Bowman, "`HAL"' zu deaktivieren.

Heute, viele Jahre später, ist das Szenario einer eigenständig denkenden Maschine präsenter denn je. Dies liegt vor allem daran, dass das seit vielen Jahren bestehende Interesse an diesem Themenbereich nun auch durch die gestiegene Rechenleistung umgesetzt werden kann. Spiele wie Schach und Go, die analytisches Denken erfordern, beherrschen Algorithmen schon lange besser als Menschen. Bilder bearbeiten, fälschen oder sogar komplett generieren ist ebenfalls keine Herausforderung mehr. Das Potential dieser Technologie ist scheinbar nur durch die eigenen Kompetenz begrenzt. \cite{M_AlphaGo_1.1} 

Umso wichtiger wird es für Entwickler, zumindest Grundwissen im Bereich "`Künstliche Intelligenz"' (auch: "`Artificial Intelligence"', im Weiteren AI) aufzubauen. Werden die vorhandenen Wissenspools im Internet betrachtet wird schnell klar, dass es eine gewaltige Menge an Informationen gibt. Viele Websites widersprechen sich in verschiedenen Themen, je nach Form der Erklärung kann ein ganz eigenes Verständnis von einem Thema aufgebaut werden. Auch wenn es unmöglich ist, diese Probleme komplett zu beseitigen, so soll durch dieses Projekt der Lernprozess im Bereich AI vereinfacht werden.

\section{Ziel der Arbeit}

Diese Arbeit dokumentiert die Entwicklung des Projekts "`The Learning Triangle"'. TLT soll ein Lernspiel, im Folgenden als \textit{Educational Game} bezeichnet, für das Thema AI darstellen. Es soll möglich sein, als Neueinsteiger mit Entwicklungsvorkenntnissen in das Thema zu finden und konkreteres Wissen für bestimmte Themenbereiche wie beispielsweise \textit{Neural Networks} zu erlangen. Dazu muss genau überlegt werden, wie das Projekt gestaltet werden kann. Ebenso wichtig ist die Lernmethodik, deren Qualität darüber entscheidet, wie gut der Anwender die Thematik erlernen kann. Als Ergebnis dieser Arbeit soll also ein Spiel entwickelt werden, welches durch ein noch auszuarbeitendes System sinnvoll Wissen im Bereich AI vermitteln kann. Die Grundlage stellt das bereits vorhandene Projekt TLT aus dem Software Engineering Kurs, welches unter Berücksichtigung der Aufgabe als Vorlage dient.

\section{Abgrenzung}

Der Bereich AI ist viel zu groß, um ihn in einem Format wie dem eines \textit{Educational Games} verpacken zu können. Deshalb ist es wichtig, klare Grenzen zu ziehen. Der Fokus dieses Projekts liegt deshalb im Groben gesehen auf dem Teilbereich \textit{Machine Learning}, dem für Softwareentwickler wichtigstem Themengebiet. Auch auf feinerer Ebene muss klar aufgezeigt werden, welche Themen bearbeitet werden. Dies wird in Kapitel \ref{lernbareTechnologien} näher erläutert. Als weitere Abgrenzung kann die Zielgruppe dieses Projektes genannt werden. Es richtet sich an Informatiker und Interessierte mit einer Wissensgrundlage in der IT. Es muss durch diese Aussage nicht ausgeschlossen werden, dass auch andere Menschen die Anwendung verwenden können, aber es soll nicht darauf geachtet, das die Erklärungen für Themen-fremde zugänglich sind.