\chapter{Fazit}
\label{fazitChapter}
In diesem Kapitel soll in Form eines Rückblickes betrachtet werden, inwiefern die Arbeit als gelungen betrachtet werden kann. Dazu wird zuerst Vorgehen, Teamentwicklung und Ergebnis evaluiert, bevor mit einem Ausblick auf zukünftige Möglichkeiten abgeschlossen wird.

\section{Evaluation der Studienarbeit}

Die Studienarbeit ist an diesem Punkt zeitlich abgeschlossen und deshalb soll ehrlich und kritisch bewertet werden, was als gut und was als schlecht anzusehen ist. Dafür wurden verschiedene Aspekte betrachtet:

\paragraph{zur Entwicklung:} Das Ergebnis in Form des Prototypes ist akzeptabel, hätte aber unter besseren Umständen einer finalen Form deutlich näher kommen können. Gründe für diesen Umstand gibt es verschiedene, als wichtigster muss zuallererst festgehalten werden, dass der Aufwand von Unity unterschätzt wurde. Tatsächlich ist es nicht schwierig, simple Programme zu erstellen. Diese sind aber weit von einem fertigen Produkt entfernt und wirken weder aufbereitet noch poliert. Mehr als ein kleines Spiel lässt sich ohne große Aufwand nicht realisieren. Des Weiteren ist sehr viel Zeit in Planung und Experimente geflossen. Als festgelegt wurde, dass Unity das Mittel der Wahl wird, ist bereits in verschiedenen Technologien mehr oder weniger viel entstanden, welches aber an Wichtigkeit verlor. Dem höheren Zweck einer wissenschaftlichen Arbeit, dem Sammeln von Erfahrung und dem Erforschen von Möglichkeiten hat dies jedoch nicht geschadet. Einzig die in der Arbeit entstandene Anwendung leidet unter der nicht optimalen Ausnutzung der gegebenen Zeit. Deshalb lässt sich zusammenfassend sagen, dass falls in Zukunft weiter daran gearbeitet werden sollte, dann müsste noch viel getan werden, um zu einem fertigen Produkt zu gelangen. 

\paragraph{zur Teamarbeit:} Die Studienarbeit wurde von zwei Personen erstellt. Das bringt mit sich, dass über die weiteren Schritte kommuniziert werden muss und dass Entscheidungen getroffen werden. Als Einzelperson lässt sich dieser Vorgang ohne Aufwand umsetzen, in einer Gruppe wird das jedoch in manchen Situationen zu einer Herausforderung. Je nach Team muss über Entscheidungen lange diskutiert werden. Die Arbeit in einer Gruppe von zwei Personen war in diesem Fall jedoch ein angenehmer Vorgang. Bis auf grundlegende Festlegungen zu Technologie und Methodik war das Vorgehen meistens klar und bedarf keiner langen Diskussion. Der Vorteil von mehr Arbeitskraft konnte deshalb zufriedenstellend ausgenutzt werden. Mehr Personen die mitarbeiten hätten einen höheren Kommunikationsaufwand bedeutet, weniger Personen hätten ein schlechteres Ergebnis zur Folge gehabt. Abschließend lässt sich also feststellen dass die Aufgabe gut für eine Teamarbeit geeignet war.

\paragraph{zur agilen Vorgehensweise:} Zum Beginn der Arbeit erschien der Aufwand, der durch das agile Vorgehen verursacht wurde unverhältnismäßig groß. Ein Backlog musste erarbeitet werden, ein Tool zur ordentlichen Verwaltung musste ausgewählt werden und sämtliche Stories und Tasks mussten eingetragen werden. Dies verlief nicht immer so automatisiert wie es von verschiedenen Aussagen versprochen wird. Nach einigen Wochen hatte sich jedoch schnell eine gewisse Routine in der Arbeit mit der agilen Methode eingespielt, sodass der zusätzliche Aufwand akzeptabel klein wurde. Zusätzliche Gründe für den doch eher ungewöhnlichen Aufwand zu Beginn waren unter anderem eine gewisse Unerfahrenheit, was häufig Zeit kostete. Außerdem musste eine Person des Teams für eine Wahlpflichtveranstaltung zusätzlich Dokumente anfertigen, um in dem Fach die Prüfung zu bestehen, wurde auch dadurch Zeit beansprucht. Nachdem diese Prüfungsleistung erbracht war und die gewissen Startschwierigkeiten beseitigt waren, half die agile Methode aber geordneter und vor allem auch disziplinierter an der Thematik zuarbeiten. Es hat geholfen, verschiedene Aufgaben aufzuteilen zu können, da dadurch weniger unerwartetes passierte. Zusammenfassend ist es sinnvoll, darüber nachzudenken der agilen Methode eine Chance zu geben, wäre auf dieses Projekt bezogen aber absehbar gewesen dass der Aufwand zu Beginn so groß ist, wäre die Entscheidung sehr wahrscheinlich anders ausgefallen.

\paragraph{zum in Kapitel \ref{timetrackingParagraph} erstellten Timetracking:} Als Gesamtaufwand für die Arbeit kann eine Zeitmenge von 165 Stunden festgehalten werden. Dabei gibt es Unterschiede zwischen den am Projekt beteiligten Personen von 20 Stunden. Gründe dafür sind zum einen unterschiedliche Wissensgrundlagen und unterschiedliche Arbeitsmethodiken während der Entwicklung. Die in vier Bereiche kategorisierten Zeiten zeigen, dass bis auf größere Unterschiede in der Entwicklung keine nennenswerten Differenzen existieren. Allgemeine Aussagen lassen sich im Anschluss jedoch trotzdem nicht treffen, da sehr situationsbezogen geurteilt werden muss. Mehr Entwicklungszeit könnte schlechteres Vorwissen, aber auch gründlichere Arbeit als Ursache haben. Außerdem ist nur mittelmäßig erkennbar, ob wirklich alle Tasks passend kategorisiert wurden und Zeiten korrekt eingeflossen sind. Deshalb soll an dieser Stelle kein Urteil gefällt werden. Stattdessen soll in kurzer Ausführung analysiert werden, wie während der Arbeit vorgegangen wurde. Dabei zeigt sich, dass in Summe die meiste Zeit für das Schreiben der Arbeit aufgebracht wurde, was keine Überraschung ist. Der Aufwand, 70 Seiten zu produzieren überwiegt in diesem Fall den, der für das Erstellen eines Prototypes notwendig war. Nicht zu verachten ist jedoch auch der Aufwand der für die Vorarbeit aufgewendet werden musste. Wünschenswerterweise  hätte weniger Zeit dafür reichen müssen, macht das Schaffen der Grundlage in dieser Arbeit doch einen Gesamtaufwand von fast einem Viertel aus. Ebenfalls lässt sich zeigen, dass sehr wenig getestet wurde, was unter anderem auf die Wahl der Technologie und die Ressourcenverteilung zurückgeführt werden kann. In einem echten Projekt hätte dafür auf jeden Fall mehr Zeit investiert werden müssen. Im Großen und Ganzen belegt das Aufschlüsseln der Zeiten die im Fazit bisher genannten Punkte sehr gut. Im Anhang findet sich außerdem noch ein weiteres Diagramm (Abb. \ref{RTimeReport}), welches ebenfalls dort erläutert wird.

\paragraph{zum in Kapitel \ref{riskmanagementchapter} erstellten Risikomanagement:} Das Risikomanagement welches zu Beginn der Arbeit ausgearbeitet wurde hat immer vor Augen geführt, inwiefern es um die möglichen Risiken steht. Grundvoraussetzung dafür war allerdings, dass es auch immer betrachtet und angepasst wurde. Dies wurde während der Studienarbeit nur unregelmäßig getan. Trotzdem sind keine bedrohlichen Risiken aufgetreten, wobei das Wissensproblem deutlich zu spüren war. Es hat viel Zeit gekostet, sich einzulesen. Die Klausuren wurden weitestgehend über das ganze Semester verteilt, sodass eigentlich immer etwas zu tun war. Das war angenehm im Bezug auf Klausuren, allerdings wurde dadurch in Summe weniger in die Studienarbeit investiert. Hardwareprobleme traten keine auf und auch krankheitsbedingt gab es keine Ausfälle. Die Kommunikation war nicht immer einfach, da verschiedene Ansichten hin und wieder zu einer Kollision der Teamkollegen führte. Es gab aber auf die ganze Arbeit gesehen keine großen Schwierigkeiten aufgrund dessen. Das Risiko mit dem größten Einfluss war mit Sicherheit die Zielsetzung, welche zu groß dimensioniert war. Nur mit einem erheblichen Mehraufwand über die angedachte Menge an Arbeit hinaus wäre es gelungen, ein finales Produkt zu entwickeln. 

Die Methode Risikomanagement an sich erfüllte ihren Zweck und sollte auch in zukünftigen Projekten verwendet werden. Mit einer regelmäßigeren Bearbeitung der Risiken würde das Vorgehen zusätzlich an Nutzen gewinnen. Im Anhang findet sich eine Tabelle die für jedes Risiko noch einmal genauer zeigt, ob es eingetreten ist und wie damit umgegangen wurde.

\paragraph{zu sonstigen nennenswerten Dingen:}
Zu Beginn wirkte der zeitliche Rahmen von sechs Monaten sehr lang, da bisherige Arbeiten mit ähnlichem Umfang nur drei Monate zugesprochen bekamen. Natürlich muss beachtet werden dass im Gegensatz zu den anderen Arbeiten nicht täglich acht Stunden dafür verwendet werden können. Aufgrund von Vorlesungen und Klausuren stellte sich am Ende heraus dass eigentlich deutlich weniger Zeit zur Verfügung stand. Im Großen und Ganzen kann das Ergebnis für die vorhandene Zeit als zufriedenstellend bewertet werden. Bereits im Bereich Entwicklung wurde über die Bewertung des Ergebnisses gesprochen.
Ebenfalls schwierig stellte sich das Anlesen und Herausschreiben der Thematik AI heraus. Viele verschiedene Auslegungen machten es nicht einfach, eine klare Meinung zu vertreten, die unter gutem Gewissen in einem Lernspiel eingebaut werden konnte. Oft musste viel Zeit für die Suche und Anpassung einer gefundenen Quelle aufgebracht werden, nachdem auch darüber diskutiert wurde. Das hätte mit einem klarer definierten Thema weniger Schwierigkeiten zur Folge gehabt
Ein wenig schade ist die Tatsache, dass viele Tätigkeiten wie beispielsweise das Anlesen von Wissen oder Ausprobieren von verschiedenen Dingen nicht aus der sichtbaren Lösung erkenntlich werden. Das lässt die Euphorie über den Abschluss der Arbeit etwas nüchterner ausfallen.

\section{Ausblick}

Als Abschluss der Arbeit soll darüber nachgedacht werden, wie dieses Projekt weitergehen würde, wenn eine Weiterarbeit zustande kommt. Dabei gibt es drei Themen die angegangen werden sollten:

\begin{itemize}
\item Der Prototyp
\item Die Lernthemen
\item Der Lernprozess
\end{itemize}

Der Prototyp ist wie der Name schon sagt nicht mehr als eine allererste Version. Dieser müsste deshalb unbedingt verbessert und erweitert werden. Dabei ist es nicht sicher, ob die Wahl der Technologie für die Zukunft die Richtige ist. Es würde jedenfalls sehr viel Zeit beanspruchen, hier eine finalen Version zu erstellen.

Die Lernthemen sind eine beschränkte Auswahl dessen, was wichtig ist und gelehrt werden könnte. Nicht nur die Menge kann hierbei verbessert werden, sondern auch wie tief in bereits aufgenommene AI-Themen eingestiegen werden könnte. Als Beispiel kann hier die Erklärung von Neuronalen Netzen aufgeführt werden, welche grundlegend die Idee und die Funktionsweise dieser Technologie aufführt. Wie genau allerdings die Mathematik eine Rolle spielt und welche Formeln wichtig werden, ist nicht beschrieben worden. Die Entscheidung dafür liegt in der kleinen Zielgruppe die mit solchen Erklärungen angesprochen werde würde. Für die weitere Entwicklung ist es aber vorstellbar, auch solche Erklärung mit aufzunehmen.

Als letzter großer Bereich der noch weiter bearbeitet werden kann ist der Lernprozess zu nennen, mit dem unterrichtet wird. Da alle Entscheidungen hierzu von Menschen getroffen wurden, die noch nie unterrichtet haben kann nicht sichergestellt werden, dass der Lernprozess optimal ist. Zielgruppentests oder das Miteinbeziehen von Fachkundigen wäre hierbei der nächste Schritt.

Im Großen und Ganzen kann nicht vorhergesehen werden, wie dieses Projekt in der Zielgruppe Informatik angenommen werden würde, da zum Abschluss des Projekts an dieser Stelle keine vorzeigbare Software entstanden ist. Nur die auf dem Papier beschriebenen Prozesse zu zeigen würde nicht garantieren, dass die Theorie auch in der Praxis eines Educational Game funktionieren würde. Zusammenfassend lässt sich das Projekt mit der Aussage abschließen, dass für weitere Arbeiten eine gute Grundlage gelegt wurde, sodass bis auf die Umsetzung mit kommerziellem Ziel alle nötigen Arbeiten zur Erstellung eines Educational Games bereits in einem ausreichendem Umfang durchgeführt wurden.