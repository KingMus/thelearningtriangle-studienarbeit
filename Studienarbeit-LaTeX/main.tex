\documentclass[a4paper]{article}

%% Language and font encodings
\usepackage[english]{babel}
\usepackage[utf8x]{inputenc}
\usepackage[T1]{fontenc}

%% Sets page size and margins
\usepackage[a4paper,top=3cm,bottom=2cm,left=3cm,right=3cm,marginparwidth=1.75cm]{geometry}

%% Useful packages
\usepackage{amsmath}
\usepackage{graphicx}
\usepackage[colorinlistoftodos]{todonotes}
\usepackage[colorlinks=true, allcolors=blue]{hyperref}

\title{The Learning Triangle - An educational game for artificial intelligence}
\author{Franziska Neumann and Marco Mueller}

\begin{document}
\maketitle

\begin{abstract}
Formatting will be done later...

Dies ist eine grobe Überlegung, ein detailliertes Inhaltsberzeichnis wird sich erst während der Arbeit herauskristallisieren. Im Bereich "Grundlagen" kann beispielsweise noch keine präzise Struktur definiert werden!
\end{abstract}

\newpage
\tableofcontents
\newpage

\section{Einleitung}

\subsection{Trendthema Artificial Intelligence}

\subsection{Ziel der Arbeit}

\subsection{Abgrenzung}

\section{Projektmanagement}

hier Erklärung zu PM-Tools, Planung, Meilensteinen...

Überarbeitung der bisher vorhandenen SE-Dokumente?

\section{Grundlagen}

hier wird noch mehr folgen

\subsection{Was ist TheLearningTriangle (TLT)}

\subsection{Historie des Projektes TLT}

\subsection{Was ist ein Educational Game}

\subsection{Was ist AI}

hier werden die für die Arbeit notwendigen Bereiche des Themas Artificial Intelligence erläutert...

\section{Vorüberlegungen zur Gestaltung von TLT}

\subsection{Verwendete Technologien zur Erstellung von TLT}

plus Erklärung warum wir uns dafür entschieden haben

\subsection{Welche Technologien können mit TLT gelernt werden}

plus Erklärung warum wir uns dafür entschieden haben

\subsection{Die Lernmethodik in TLT}

wie kann man das Thema gut beibringen bzw wie kann TLT gestaltet werden um sinnvoll zu lehren...

\subsection{Strukturierung von TLT}

Gliederung? (Lektionen, Level ...)

Motivation? (Gamification, ...)

UI?

\section{Umsetzung}

ziemlich offen, folgt später detaillierter...

beinhaltet: Architektur, Klassendiagramm und weitere ähnliche wichtige Dinge

\section{Fazit}

\subsection{Evaluation der Studienarbeit}

was wurde erreicht

Zielgruppenbefragung?

\subsection{Ausblick}

\section{Literatur}

\section{Anhang}

\bibliographystyle{alpha}
\bibliography{sample}

\end{document}